%*----------- SLIDE -------------------------------------------------------------
\begin{frame}[c]{}

    \centering
    \includegraphics[width=.85\textwidth, trim= 0 30 0 0, clip]{revisão bibliogrática Diagram.png}
    
        
    % \begin{tikzpicture}[remember picture,overlay]
    %     \draw[red,very thick] (cmark) circle[x radius=8mm,y radius=4mm]; 
    % \end{tikzpicture}
%*----------- notes
    \note[item]{Notes can help you to remember important information. Turn on the notes option.}
\end{frame}
%-
%*----------- SLIDE -------------------------------------------------------------
\begin{frame}[t]{O progresso das equipes}
    Um dos indicadores para o acompanhamento das equipes será o percentual de conclusão geral da equipe.
    O planejamento das atividades deverá seguir a metodologia aplicada no desenvolvimento de projetos de robótica.
    \newline
    %\vspace{0.5cm}
    \begin{table}[ht!]
    \centering
        \caption{PERCENTUAL DE CONCLUSÃO POR EQUIPE}
        \begin{tabular}{|l|c|c|c|c|} \hline
            \textbf{EQUIPE}&\textbf{04/05}&\textbf{11/05}&\textbf{18/05}&\textbf{25/05}\\ \hline
            RAJA & 17\% &32\% & &  \\ \hline
            BORG & 0\% &41\% & &  \\ \hline
            TIMON-HM & 5\% &47\% & &  \\ \hline
        \end{tabular}
    \end{table}
%*----------- notes
    \note[item]{Notes can help you to remember important information. Turn on the notes option.}
\end{frame}
%-